\chapter*{Заключение}						% Заголовок
\addcontentsline{toc}{chapter}{Заключение}	% Добавляем его в оглавление

%% Согласно ГОСТ Р 7.0.11-2011:
%% 5.3.3 В заключении диссертации излагают итоги выполненного исследования, рекомендации, перспективы дальнейшей разработки темы.
%% 9.2.3 В заключении автореферата диссертации излагают итоги данного исследования, рекомендации и перспективы дальнейшей разработки темы.
%% Поэтому имеет смысл сделать эту часть общей и загрузить из одного файла в автореферат и в диссертацию:


В рамках диссертационной работы проведены исследования влияния изовалентного замещения в соединениях Cu\textsubscript{12}As\textsubscript{4}S\textsubscript{13}, Cu\textsubscript{12}Sb\textsubscript{4}S\textsubscript{13}, Cu\textsubscript{3}AsSe\textsubscript{3} и Cu\textsubscript{3}SbSe\textsubscript{3} на их транспортные и магнитиные свойства. 
В работе проведено исследование  синтетического теннантита Cu\textsubscript{12}As\textsubscript{4}S\textsubscript{13} в диапазоне температур от 85 до 300~К методами монокристальной рентгеновской дифрактометрии, высокоразрещающей микроскопии и комбинационного рассеяния света в монокристаллическом образце.

Основные результаты работы заключаются в следующем:
%% Согласно ГОСТ Р 7.0.11-2011:
%% 5.3.3 В заключении диссертации излагают итоги выполненного исследования, рекомендации, перспективы дальнейшей разработки темы.
%% 9.2.3 В заключении автореферата диссертации излагают итоги данного исследования, рекомендации и перспективы дальнейшей разработки темы.
\begin{enumerate}
						\item На основе анализа экспериментальных данных, полученных в рентгеноструктурных температурных экспериментах, элекронной микроскопии и на основе результатов квантомеханического моделирования, установлено, что
						лавесовский полиэдр сформирован шестью атомами меди, которые лежат в неэквивалентных позициях Cu2 и Cu21, а на структурную формулу синтетического теннантита Сu\textsubscript{12}As\textsubscript{4}S\textsubscript{13} приходится 12 атомов меди.
						\item Через сравнение данных расчётной  и  экспериментальной теплоёмкостей, полученной сканирующей дифференциальной калориметрией и спектроскопии комбинационного рассеяния света при комнатной температуре, показано влияние ассиметричной связи As(CuS\textsubscript{3})As, возникающей ввиду наличия неэквивалентных позициий Cu2 и Cu21, на размягчение фононных мод в структуре синтетического теннантита Сu\textsubscript{12}As\textsubscript{4}S\textsubscript{13}.
						\item Методами рентгеноструктурного анализа, сканирующей дифференциальной калориметрии, магнитометрии и первопринципных расчётов обнаружен фазовый переход второго рода в синтетическом теннантите Сu\textsubscript{12}As\textsubscript{4}S\textsubscript{13} при температуре 124 К.
						\item Методами сканирующей дифференциальной калориметрии, магнитометрии и спектроскопии комбинационного рассеяния света выявлены  низкоэнергетические фононные моды (ввиду  ассиметричной связи в (As,Sb)(Cu(S, Se)\textsubscript{3})(As,Sb)) в  соединениях из группы тетраэдритов-теннантитов. Найденные значения энергий фононных мод лежат в диапазоне от 5 до 40~мэВ.
						\item На основе результатов работы и опубликованных литературных данных рассмотрено влияние изовалентного замещения в Cu--(As,Sb)--S на значения температур фазовых переходов второго рода.

\end{enumerate}


Результаты работы и литературные данные показывают, что локальное окружение меди влияет на физические свойства (в частности на магнитные и транспортные). Помимо полученных фундаментальных результатов данная работа, вносит вклад в развитие современного материаловедения и дополняет существующие представления о механизмах формирования физических свойств сложных сульфосолей.



