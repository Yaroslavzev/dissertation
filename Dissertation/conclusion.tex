\chapter*{Заключение}						% Заголовок
\addcontentsline{toc}{chapter}{Заключение}	% Добавляем его в оглавление

%% Согласно ГОСТ Р 7.0.11-2011:
%% 5.3.3 В заключении диссертации излагают итоги выполненного исследования, рекомендации, перспективы дальнейшей разработки темы.
%% 9.2.3 В заключении автореферата диссертации излагают итоги данного исследования, рекомендации и перспективы дальнейшей разработки темы.
%% Поэтому имеет смысл сделать эту часть общей и загрузить из одного файла в автореферат и в диссертацию:


В рамках диссертационной работы проведены исследования влияния изовалентного замещения в соединениях Cu\textsubscript{12}As\textsubscript{4}S\textsubscript{13}, Cu\textsubscript{12}Sb\textsubscript{4}S\textsubscript{13}, Cu\textsubscript{3}AsSe\textsubscript{3} и Cu\textsubscript{3}SbSe\textsubscript{3} на их транспортные и магнитиные свойства. 
В работе проведено исследование  синтетического теннантита Cu\textsubscript{12}As\textsubscript{4}S\textsubscript{13} в диапазоне температур от 85 до 300~К методами монокристальной рентгеновской дифрактометрии, высокоразрещающей микроскопии и комбинационного рассеяния света в монокристаллическом образце.

Основные результаты работы заключаются в следующем:
% %% Согласно ГОСТ Р 7.0.11-2011:
%% 5.3.3 В заключении диссертации излагают итоги выполненного исследования, рекомендации, перспективы дальнейшей разработки темы.
%% 9.2.3 В заключении автореферата диссертации излагают итоги данного исследования, рекомендации и перспективы дальнейшей разработки темы.
\begin{enumerate}
						\item Рассмотрены перспективные для создания новых функциональных материалов неорганические соединения солей из группы теннантита"--~тетраэдрита, в которых наблюдается переменная валентность на атомах одного сорта;
						\item Результаты анализа формирования лавесовского полиэдра впервые выявили схожие структурные зависимости для крайних членов солей из группы теннантита"--~тетраэдрита;
						\item Показано, что на структурную формулу в синтетическом теннантите Сu\textsubscript{12}As\textsubscript{4}S\textsubscript{13} приходится 12 атомов меди;
						\item Проведен анализ расположения атомов меди в лавесовском полиэдре методом первопринципных расчетов, показано наличие рядов меди с распределением электронной плотности в форме эллипса для плоскости (011) атомарного изображения структуры синтетического теннантита Сu\textsubscript{12}As\textsubscript{4}S\textsubscript{13};
						\item Обнаружен фазовый переход 2 рода при температуре 124 К в синтетическом теннантите Сu\textsubscript{12}As\textsubscript{4}S\textsubscript{13} методами монокристальной дифрактометрии, релаксации теплового импульса и  первопринципных расчетов;
						\item Показано возникновение низкоэнергетических фононных мод (ввиду  ассиметричной связи в As(CuS\textsubscript{3})As) в соединениях синтетических теннантита Сu\textsubscript{12}As\textsubscript{4}S\textsubscript{13} и мгриита Cu\textsubscript{3}AsSe\textsubscript{3} путём сравнения расчетной и экспериментальной зависимостей теплоёмкостей. Определенные энергии  фононных мод  лежат в диапазоне от 5 до 40~мэВ;
						\item Получены экспериментальные значения энергий низкоэнергетических фононных мод для исследуемых соединений, которые лежат в диапазоне от 5 до 25~мэВ;
						\item Изовалентное замещение Cu--(As,Sb)--S приводит к изменению температуры Нееля с $\approx$124 К на $\approx$84 К.

\end{enumerate}

\begin{enumerate}
\item Показано, что синтетический теннантит Сu\textsubscript{12}As\textsubscript{4}S\textsubscript{13} обладает неэквивалентным  окружением атомов меди в позициях Cu21 и Cu2, а в элементарной ячейке содержит 12 атомов меди.
\item Неэквивалентные позиции атомов меди Cu21 и Cu2 в синтетическом теннантите способствуют появлению ассиметричных связей As(CuS\textsubscript{3})As, которые приводят к размягчению фононных мод в структуре синтетического теннантита Сu\textsubscript{12}As\textsubscript{4}S\textsubscript{13}.
\item На основе данных аномального изменения значения коэффициента атомарного смещения для позиции S2 при понижении температуры, по скачкобразному изменению теплоёмкости при 124~K, полученной сканирующей дифференциальной калориметрией, по отклонению температурной зависимости намагниченности от закона Кюри--Вейсса при $\approx$~124~K и по сравнению энергий кристаллической решётки для структур при 85~K и 293~К, полученных методом первопринципных расчётов, в синтетическом теннантите Сu\textsubscript{12}As\textsubscript{4}S\textsubscript{13} при температуре 124~К обнаружен фазовый переход второго рода.
\item В соединениях из группы тетраэдритов-теннантитов экспериментально выявлены низкоэнергетические фононные моды в диапазоне от 5 до 40~мэВ, наличие которых обусловлено  ассимметричными связями (As,Sb)(Cu(S, Se)\textsubscript{3})(As,Sb).
\item Показано, что изовалентное замещение Cu--(As,Sb)--S приводит к изменению локального окружения атомов меди и к понижению температуры Нееля с $\approx$124 К до $\approx$84 К.

\end{enumerate}

Результаты работы и литературные данные показывают, что локальное окружение меди влияет на физические свойства (в частности на магнитные и транспортные). Помимо полученных фундаментальных результатов данная работа, вносит вклад в развитие современного материаловедения и дополняет существующие представления о механизмах формирования физических свойств сложных сульфосолей.



