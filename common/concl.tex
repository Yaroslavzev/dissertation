%% Согласно ГОСТ Р 7.0.11-2011:
%% 5.3.3 В заключении диссертации излагают итоги выполненного исследования, рекомендации, перспективы дальнейшей разработки темы.
%% 9.2.3 В заключении автореферата диссертации излагают итоги данного исследования, рекомендации и перспективы дальнейшей разработки темы.
\begin{enumerate}
						\item Рассмотрены перспективные для создания новых функциональных материалов неорганические соединения солей из группы теннантита"--~тетраэдрита, в которых наблюдается переменная валентность на атомах одного сорта;
						\item Результаты анализа формирования лавесовского полиэдра впервые выявили схожие структурные зависимости для крайних членов солей из группы теннантита"--~тетраэдрита;
						\item Показано, что на структурную формулу в синтетическом теннантите Сu\textsubscript{12}As\textsubscript{4}S\textsubscript{13} приходится 12 атомов меди;
						\item Проведен анализ расположения атомов меди в лавесовском полиэдре методом первопринципных расчетов, показано наличие рядов меди с распределением электронной плотности в форме эллипса для плоскости (011) атомарного изображения структуры синтетического теннантита Сu\textsubscript{12}As\textsubscript{4}S\textsubscript{13};
						\item Обнаружен фазовый переход 2 рода при температуре 124 К в синтетическом теннантите Сu\textsubscript{12}As\textsubscript{4}S\textsubscript{13} методами монокристальной дифрактометрии, релаксации теплового импульса и  первопринципных расчетов;
						\item Показано возникновение низкоэнергетических фононных мод (ввиду  ассиметричной связи в As(CuS\textsubscript{3})As) в соединениях синтетических теннантита Сu\textsubscript{12}As\textsubscript{4}S\textsubscript{13} и мгриита Cu\textsubscript{3}AsSe\textsubscript{3} путём сравнения расчетной и экспериментальной зависимостей теплоёмкостей. Определенные энергии  фононных мод  лежат в диапазоне от 5 до 40~мэВ;
						\item Получены экспериментальные значения энергий низкоэнергетических фононных мод для исследуемых соединений, которые лежат в диапазоне от 5 до 25~мэВ;
						\item Изовалентное замещение Cu--(As,Sb)--S приводит к изменению температуры Нееля с $\approx$124 К на $\approx$84 К.

\end{enumerate}
