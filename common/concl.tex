%% Согласно ГОСТ Р 7.0.11-2011:
%% 5.3.3 В заключении диссертации излагают итоги выполненного исследования, рекомендации, перспективы дальнейшей разработки темы.
%% 9.2.3 В заключении автореферата диссертации излагают итоги данного исследования, рекомендации и перспективы дальнейшей разработки темы.
\begin{enumerate}
						\item Показано, что на структурную формулу в синтетическом теннантите Сu\textsubscript{12}As\textsubscript{4}S\textsubscript{13} приходится 12 атомов меди.
						\item Показано влияние ассиметричной связи As(CuS\textsubscript{3})As на размягчение фононных мод в структуре синтетического теннантита Сu\textsubscript{12}As\textsubscript{4}S\textsubscript{13}.
						\item Обнаружен фазовый переход второго рода при температуре 124 К в синтетическом теннантите Сu\textsubscript{12}As\textsubscript{4}S\textsubscript{13} методами монокристальной дифрактометрии, релаксации теплового импульса и  первопринципных расчётов.
						\item Показано возникновение низкоэнергетических фононных мод (ввиду  ассиметричной связи в As(CuS\textsubscript{3})As) в соединениях синтетических теннантита Сu\textsubscript{12}As\textsubscript{4}S\textsubscript{13} и мгриита Cu\textsubscript{3}AsSe\textsubscript{3} путём сравнения расчётной и экспериментальных зависимостей теплоёмкостей. Определённые энергии  фононных мод  лежат в диапазоне от 5 до 40~мэВ.
						\item Показано влияние изовалентного замещения в Cu--(As,Sb)--S и  Cu--(As,Sb)--Se на значения температурных диапазонов  особых магнитных состояний.

\end{enumerate}
