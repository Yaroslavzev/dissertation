{\actuality} В настоящее время известно большое число исследований магнетизма и сверхпроводимости соединений с атомами переходных металлов\cite{Slocombe_2015}.
Неорганические соединения, включающие в себя атомы с 3d\textsuperscript{9} электронной оболочкой и обладающие несимметричной поверхностью Ферми, такие как  керамические соединения купратов, тонкопленочные структуры  и монокристаллы с атомами интерметаллидов, представляют интерес для исследований и обладают практической ценностью как функциональные материалы.

Одним из актуальных исследований является работа \cite{Blandy_2018}, в которой экспериментально показано возникновение антиферромагнитного упорядочения в слоистой структуре Sr\textsubscript{2}CuO\textsubscript{2}Cu\textsubscript{2}S\textsubscript{2} на атомах меди Cu\textsuperscript{+} и Cu\textsuperscript{2+}.
Так, например, в работе, посвященной сверхпроводящим купратам\cite{Comin_2015}, отмечается высокая вероятность возникновения спин-флуктуационного механизма, который приводит к искажению поверхности Ферми и возникновению d\textsubscript{x\textsubscript{2}-y\textsubscript{2}}-симметрии параметра порядка вблизи атомов меди.
Наблюдаемые искажения приводят к возникновению переменной валентности на атомах меди и обусловлены особенностями кристаллической структуры: переходный металл находится в слоях между полиэдрами разной конфигурации, а образованные полиэдрами пустоты заполняются стабилизирующими структуру катионами.
Таким образом, создаются структурные комбинации, которые приводят к возникновению сложной электронной структуры.


Другими перспективными для исследования неорганическими материалами, в которых наблюдается переменная валентность на атомах одного сорта, являются соединения солей из группы тетраэдрита-теннантита.
В основе этих соединений лежит лавесовский полиэдр: усеченный тетраэдр (Рис.~\ref{img:figure1}).
Полиэдр сформирован 6 атомами меди, которые располагаются  в центре рёбер не усеченного тетраэдра.
Сформированный лавесовский полиэдр окружен тетраэдрическими комплексами, направленными в одну сторону.
  В такой структуре тетраэдритов и теннантитов, ввиду ян-тейллоровского искажения,   наблюдается сдвиг атомов меди из идеального положения, который приводит к  перекрытию электронных орбиталей атомов меди и формирует особенную электронную структуру.
Высокая чувствительность электронной структуры к расстоянию между атомами меди и чувствительность к окружению лавесовского полиэдра  делает соединения из группы тетраэдрита--теннантита  интересными для исследований и прикладных применений.

В современной литературе эти соединения рассматриваются как функциональные материалы для термоэлектриков ввиду их низкой теплопроводности, вызванной наличием изолированных комплексов в структуре.
Некоторые соединения из этой группы обладают магнитными свойствами, которые возникают благодаря сложным ионно--ковалентным связям. В некоторых работах отмечается возможное магнитное упорядочение в соединениях теннантита и тетраэдрита.
Также преимущественно сульфосоли являются индикаторами золотосодержащих руд, потребность в которых высока.

В данной работе представлено исследование влияния изовалентного замещения в соединениях Cu\textsubscript{12}As\textsubscript{4}S\textsubscript{13}, Cu\textsubscript{12}Sb\textsubscript{4}S\textsubscript{13}, Cu\textsubscript{3}AsSe\textsubscript{3} и Cu\textsubscript{3}SbSe\textsubscript{3} на транспортные и магнитные свойства этих соединений.
Также в работе приведено рентгеноструктурное исследование соединения  Cu\textsubscript{12}As\textsubscript{4}S\textsubscript{13} в диапазоне температур от 85 до 293~К.  Помимо полученных фундаментальных результатов данная работа вносит вклад в развитие современного материаловедения и дополняет существующие представления о механизмах формирования физических свойств.

 \aim\ данного исследования является определение положения атомов меди в синтетическом теннантите
 Cu\textsubscript{12}As\textsubscript{4}S\textsubscript{13}, изучение на его примере особенностей формирования лавесовского полиэдра и влияния изовалентного замещения на магнитные и транспортные свойства соединений из группы тетраэдрита--теннантита на примере соединений  Cu\textsubscript{12}As\textsubscript{4}S\textsubscript{13}, Cu\textsubscript{12}Sb\textsubscript{4}S\textsubscript{13}, Cu\textsubscript{3}AsSe\textsubscript{3} и Cu\textsubscript{3}SbSe\textsubscript{3}.

Для~достижения поставленной цели необходимо было решить следующие основные {\tasks}:
\begin{enumerate}
  \item Исследование положения атомов меди методами монокристальной рентгеновской дифрактометрии, высокоразрешающей просвечивающей  электронной микроскопии и комбинационного рассеяния света в монокристаллическом образце синтетического теннантита Cu\textsubscript{12}As\textsubscript{4}S\textsubscript{13} в диапазоне температур от 85 до 300~К.
  \item Измерение зависимостей теплоёмкости образцов Cu\textsubscript{12}As\textsubscript{4}S\textsubscript{13} и Cu\textsubscript{3}AsSe\textsubscript{3} и магнитной восприимчивости  образцов Cu\textsubscript{12}As\textsubscript{4}S\textsubscript{13}, Cu\textsubscript{12}Sb\textsubscript{4}S\textsubscript{13}, Cu\textsubscript{3}AsSe\textsubscript{3} и Cu\textsubscript{3}SbSe\textsubscript{3} в диапазонах температур от 2  до 350~К методами сканирующей дифференциальной калориметрии и магнитометрии, а также методом спектроскопии комбинационного рассеяния света при комнатной температуре.

\end{enumerate}

\defpositions
\begin{enumerate}
\item Показано, что синтетический теннантит Сu\textsubscript{12}As\textsubscript{4}S\textsubscript{13} обладает неэквивалентным  окружением атомов меди в позициях Cu21 и Cu2, а в элементарной ячейке содержит 12 атомов меди.
\item Неэквивалентные позиции атомов меди Cu21 и Cu2 в синтетическом теннантите прособствуют появлению ассиметричных связей As(CuS\textsubscript{3})As, которые приводят к размягчению фононных мод в структуре синтетического теннантита Сu\textsubscript{12}As\textsubscript{4}S\textsubscript{13}.
\item На основе данных аномального изменения значения коэффициента атомарного смещения для позиции S2 при понижении температуры, по скачкобразному изменению теплоёмкости при 124~K, полученной сканирующей дифференциальной калориметрией, по отклонению температурной зависимости намагниченности от закона Кюри--Вейсса при $\approx$~124~K и по сравнению энергий кристаллической решётки для структур при 85~K и 293~К, полученных методом первопринципных расчётов, в синтетическом теннантите Сu\textsubscript{12}As\textsubscript{4}S\textsubscript{13} при температуре 124~К обнаружен фазовый переход второго рода.
\item В соединениях из группы тетраэдритов-теннантитов экспериментально выявлены низкоэнергетические фононные моды в диапазоне от 5 до 40~мэВ, наличие которых обусловлено  ассимметричными связями (As,Sb)(Cu(S, Se)\textsubscript{3})(As,Sb).
\item Показано, что изовалентное замещение Cu--(As,Sb)--S приводит к изменению локального окружения атомов меди и к понижению температуры Нееля с $\approx$124 К до $\approx$84 К.

\end{enumerate}

\novelty
\begin{enumerate}
\item Лавесовский полиэдр сформирован шестью атомами меди, которые лежат в неэквивалентных позициях Cu2 и Cu21, а на структурную формулу синтетического теннантита Сu\textsubscript{12}As\textsubscript{4}S\textsubscript{13} приходится 12 атомов меди.
\item По результатам рентгеноструктурного анализа, анализа спектра комбинационного рассеяния света и квантовомеханических расчётов структур синтетического теннантита Сu\textsubscript{12}As\textsubscript{4}S\textsubscript{13} обнаружено влияние ассиметричных связей As(CuS\textsubscript{3})As на размягчение фононных мод в структуре синтетического теннантита Сu\textsubscript{12}As\textsubscript{4}S\textsubscript{13}.
\item В соединении  синтетического теннантита Сu\textsubscript{12}As\textsubscript{4}S\textsubscript{13} обнаружен фазовый переход второго рода
 при 124~К методами рентгеноструктурного анализа, сканирующей дифференциальной калориметрии, магнитометрии и первопринципных расчётов.
\item Методами сканирующей дифференциальной калориметрии, магнитометрии и спектроскопии комбинационного рассеяния света экспериментально выявлены низкоэнергетические фононные моды  в соединениях из группы тетраэдритов-теннантитов, наличие которых обусловлено  ассиметричными связями \sloppy{(As,Sb)(Cu(S, Se)\textsubscript{3})(As,Sb)}.
\item Обнаружено влияние изовалентного замещения в Cu--(As,Sb)--S на значения температур фазовых переходов второго рода.
\end{enumerate}

\influence\ заключается в уникальных научных данных, которые выявляют связь наблюдаемых физических характеристик с особенностями кристаллической структуры.
Результаты структурных и спектроскопических исследований показывают причины низкой теплопроводности и позволяют целенаправлено производить поиск новых функциональных соединений с характерными полиэдрами в структуре. Результаты квантовомеханических расчётов показывают причины возникновения переменной валентности в структуре синтетического теннантита и делают соединения группы тетраэдрита--теннантита интересными с точки зрения изучения магнитных свойств.
Температурные зависимости магнитной восприимчивости изовалентных аналогов синтетического теннантита и температурные исследования структуры синтетического теннантита могут быть использованы для планирования дальнейших исследований магнитных фазовых превращений в соединениях группы тетраэдрита--теннантита.

\reliability\ полученных результатов обеспечивается комплексными
исследованиями материалов  методами рентгеновской монокристальной дифрактометрии, электронной просвечивающей микроскопии, калориметрии, исследованием намагниченности и транспортных свойств и применением современного оборудования
сертифицированного в соответствии с российскими и международными стандартами.
Достоверность и высокое качество полученных результатов
подтверждается публикациями материалов работы в рецензируемых научных журналах из перечня ВАК, а также докладами на российских и международных
конференциях.
Часть полученных результатов независимо воспроизводит известные выводы, ранее полученные другими авторами.

\probation\
Основные результаты работы докладывались~на российских и международных конференциях:
\begin{itemize}
\item  XXV Российская молодежная научная конференция, посвященная 95"~летию основания Уральского университета, <<Проблемы теоретической и экспериментальной химии>>, 22"--~24 апреля 2015, г. Екатеринбург;
\item Вторая Всероссийская молодежная научно-техническая конференция с международным участием <<Инновации в материаловедении>>, 1"--~4 июня 2015, г. Москва;
\item Международная Конференция, посвященная 80"~летию чл."~кор. РАН И.К. Камилова, <<Фазовые переходы, критические и нелинейные явления в конденсированных средах>>,  20"--~28 августа 2015, г. Челябинск;
\item ХII Российская ежегодная конференция молодых научных сотрудников и аспирантов <<Физико-химия и технология неорганических материалов>>, 13"--~16 октября 2015, г. Москва;
\item III Всероссийская научная молодежная конференция
<<Актуальные проблемы нано- и микроэлектроники>>, с 30 октября по 4 декабря 2015, г.~Уфа;
\item III Международная молодёжная научная конференция: Физика. Технологии. Инновации ФТИ"--~2016, 16"--~20 мая 2016, г. Екатеринбург;
\item Международная конференция молодых ученых, работающих в области углеродных материалов, с 30 мая по 1 июня 2017, г. Троицк, г. Москва;
\item Cедьмая международная конференция <<Кристаллофизика и деформационное поведение перспективных материалов>>, посвященная памяти профессора С.С. Горелика, c 2"--~5 октября 2017, г. Москва.
\end{itemize}


%и самостоятельно сформулировал \textcolor{red}{научную проблему --- проблему чего}
\contribution\ В диссертации представлены результаты, полученные лично соискателем, либо при его личном участии. Автор диссертации принимал непосредственное участие в выборе используемых методик. В работах по теме диссертации, выполненных с соавторами, автором диссертации выполнены постановка целей и задач, анализ, обработка и обобщение полученных результатов экспериментов. Автором использованы алгоритмы обработки данных, на основе которых получены два свидетельства о государственной регистрации программ для электронно-вычислительных машин №~2018664257 от 14.11.2018 и №~2018664530 от 19.11.2018.

%\publications\ Основные результаты по теме диссертации изложены в ХХ печатных изданиях~\cite{Sokolov,Gaidaenko,Lermontov,Management},
%Х из которых изданы в журналах, рекомендованных ВАК~\cite{Sokolov,Gaidaenko},
%ХХ --- в тезисах докладов~\cite{Lermontov,Management}.

\ifthenelse{\equal{\thebibliosel}{0}}{% Встроенная реализация с загрузкой файла через движок bibtex8
    \publications\ Основные результаты по теме диссертации изложены в XX печатных изданиях,
    X из которых изданы в журналах, рекомендованных ВАК,
    X "--- в тезисах докладов.%
}{% Реализация пакетом biblatex через движок biber
%Сделана отдельная секция, чтобы не отображались в списке цитированных материалов
    \begin{refsection}%
        \printbibliography[heading=countauthornotvak, env=countauthornotvak, keyword=biblioauthornotvak, section=1]%
        \printbibliography[heading=countauthorvak, env=countauthorvak, keyword=biblioauthorvak, section=1]%
        \printbibliography[heading=countauthorconf, env=countauthorconf, keyword=biblioauthorconf, section=1]%
        \printbibliography[heading=countauthor, env=countauthor, keyword=biblioauthor, section=1]%
        \publications\ Опубликовано 16 работ в печатных изданиях, в том числе 8 работ в журналах, рекомендованных ВАК. Основные результаты по теме диссертации изложены в 12 %\arabic{citeauthor}
 печатных изданиях\nocite{vakbibl9,vakbibl8,vakbibl2,vakbibl1,vakbibl7,vakbibl5,vakbibl6,vakbibl4}, %,,conf8,conf7,conf6,conf5,conf4,conf3,conf2,conf1
       4 %\arabic{citeauthorvak}
из которых изданы в журналах, рекомендованных ВАК,
%\nocite{vakbibl2,vakbibl1,vakbibl7}, %,vakbibl5,vakbibl4,\arabic{citeauthorconf}
8 "--- в тезисах докладов.
%\nocite{conf8,conf7,conf6,conf5,conf4,conf3,conf2,conf1}.
    \end{refsection}
}
%При использовании пакета \verb!biblatex! для автоматического подсчёта
%количества публикаций автора по теме диссертации, необходимо
%их здесь перечислить с использованием команды \verb!\nocite!.
